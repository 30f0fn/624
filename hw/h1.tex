% Options for packages loaded elsewhere
\PassOptionsToPackage{unicode}{hyperref}
\PassOptionsToPackage{hyphens}{url}
%
\documentclass[
]{article}
\usepackage{lmodern}
\usepackage{amssymb,amsmath}
\usepackage{ifxetex,ifluatex}
\ifnum 0\ifxetex 1\fi\ifluatex 1\fi=0 % if pdftex
  \usepackage[T1]{fontenc}
  \usepackage[utf8]{inputenc}
  \usepackage{textcomp} % provide euro and other symbols
\else % if luatex or xetex
  \usepackage{unicode-math}
  \defaultfontfeatures{Scale=MatchLowercase}
  \defaultfontfeatures[\rmfamily]{Ligatures=TeX,Scale=1}
\fi
% Use upquote if available, for straight quotes in verbatim environments
\IfFileExists{upquote.sty}{\usepackage{upquote}}{}
\IfFileExists{microtype.sty}{% use microtype if available
  \usepackage[]{microtype}
  \UseMicrotypeSet[protrusion]{basicmath} % disable protrusion for tt fonts
}{}
\makeatletter
\@ifundefined{KOMAClassName}{% if non-KOMA class
  \IfFileExists{parskip.sty}{%
    \usepackage{parskip}
  }{% else
    \setlength{\parindent}{0pt}
    \setlength{\parskip}{6pt plus 2pt minus 1pt}}
}{% if KOMA class
  \KOMAoptions{parskip=half}}
\makeatother
\usepackage{xcolor}
\IfFileExists{xurl.sty}{\usepackage{xurl}}{} % add URL line breaks if available
\IfFileExists{bookmark.sty}{\usepackage{bookmark}}{\usepackage{hyperref}}
\hypersetup{
  hidelinks,
  pdfcreator={LaTeX via pandoc}}
\urlstyle{same} % disable monospaced font for URLs
\setlength{\emergencystretch}{3em} % prevent overfull lines
\providecommand{\tightlist}{%
  \setlength{\itemsep}{0pt}\setlength{\parskip}{0pt}}
\setcounter{secnumdepth}{-\maxdimen} % remove section numbering

\date{3 February 2020}
\title{CS 624 - HW 1}
\author{Max Weiss}
\begin{document}
\maketitle

\hypertarget{a}{%
\paragraph{1 (a)}\label{a}}

A derivation on p20 of the handout gives \begin{equation*}
  \label{eq:fib}
  \sum_{n=0}^{\infty}f_nx^n = \frac{x}{1-x-x^2}
\end{equation*} Differentiating both sides, 
\begin{align}
  \label{eq:fibdiff}
  \sum_{n=0}^{\infty}nf_nx^{n-1} = \frac{(1-x-x^2) - x(-1-2x)}{(1-x-x^2)^2}
  = \frac{1 + x^2}{(1-x-x^2)^2}.
\end{align} Substituting \(1/2\) for \(x\), \begin{align*}
  \sum_{n=0}^{\infty}\frac{nf_n}{2^{n-1}}
  = \frac{1 + \frac{1}{4}}{(1-\frac{1}{2}-\frac{1}{4})^2}
  = \frac{\frac{5}{4}}{\frac{1}{16}}
    = 20.
\end{align*}

\hypertarget{b}{%
\paragraph{1 (b)}\label{b}}

Substituting \(1\) for \(x\) in (\ref{eq:fibdiff}) would appear to give
\begin{align}
  \label{eq:fibdiffconscrazy}
  \sum_{n=0}^{\infty}nf_n = \frac{(1-1-1) - 1(-1-2)}{(1-1-1)^2} = 2.
\end{align} However, (\ref{eq:fibdiffconscrazy}) is false; indeed, the
sum on the left hand side does not converge at all. For, toward a
contradiction, suppose that the sum converges to \(d\). Since \(f_n>n\)
for all \(n>3\) (as is clear by induction on \(n\)), it follows that
\(nf_n>n\) for all \(n>3\). Thus, \(\sum_{n=0}^{e}nf_n > d+1\) for any
\(e\) greater than both \(d\) and \(3\). So, any partial sum of length
at least \(e\) cannot come within even \(\epsilon=1\) of \(d\),
contradicting the definition of convergence.

\hypertarget{a-1}{%
\paragraph{2(a)}\label{a-1}}

According to the binomial theorem, \begin{equation}
  \label{eq:binthm}
  (1+x)^n = \sum_{k=0}^{n}\binom{n}{k}x^k.
\end{equation}

\hypertarget{b-1}{%
\paragraph{2(b)}\label{b-1}}

Similarly, \begin{equation}
  \label{eq:binthm2}
  (1-x)^n = \sum_{k=0}^{n}\binom{n}{k}(-x)^k.
\end{equation}

\hypertarget{c}{%
\paragraph{2(c)}\label{c}}

Using (\ref{eq:binthm}) and (\ref{eq:binthm2}) above, \begin{align*}
  (1 + x)^n + (1 - x)^n
  &= \sum_{k=0}^{n}\binom{n}{k}x^k + \sum_{k=0}^{n}\binom{n}{k}(-x)^k\\
  &= \sum_{k=0}^{n}\binom{n}{k}(x^k + (-x)^k)\\
  &= 2\left(\binom{n}{0}x^0 + \binom{n}{2}x^2 + \binom{n}{4}x^4 + \cdots\right).
\end{align*}

\hypertarget{d}{%
\paragraph{2(d)}\label{d}}

Likewise using (\ref{eq:binthm}) and (\ref{eq:binthm2}) above,
\begin{align}
\label{eq:bincons}
  \nonumber (1 + x)^n - (1 - x)^n
  &= \sum_{k=0}^{n}\binom{n}{k}x^k - \sum_{k=0}^{n}\binom{n}{k}(-x)^k\\
  \nonumber
  &= \sum_{k=0}^{n}\binom{n}{k}(x^k - (-x)^k)\\
  &= 2\left(\binom{n}{1}x^1 + \binom{n}{3}x^3 + \binom{n}{5}x^5 + \cdots\right).
\end{align}

\hypertarget{e}{%
\paragraph{2(e)}\label{e}}

Setting \(x=1\), the equation (\ref{eq:bincons}) implies
\begin{equation*}
  2^n = (1 + 1)^n - (1 - 1)^n = 2\left(\binom{n}{1} + \binom{n}{3} + \binom{n}{5} + \cdots\right)
\end{equation*} and so \begin{equation*}
   \binom{n}{1} + \binom{n}{3} + \binom{n}{5} + \cdots = 2^{n-1}
\end{equation*} which should be simple enough for the econ prof!

\hypertarget{a-2}{%
\paragraph{3 (a)}\label{a-2}}

Let me write \(f(n) = n^2\) and \(g(n) = 2^n\). Note that
\begin{equation*}
f(n+1) = (n+1)^2 = n^2 + 2n + 1 = f(n) + 2n + 1  
\end{equation*} while \begin{equation*}
g(n+1)=2^{n+1}=g(n)+g(n).
\end{equation*} It follows that if \(g(n)>f(n)\) and \(g(n)>2n+1\), then
\(g(n+1)>f(n+1)\).

I'll argue that \(g(n) > 2n+1\) for all \(n\geq 5\). This is clear for
\(n=5\). Now assume that \(g(n)>2n+1\). Then \begin{equation*}
g(n+1) = 2g(n) > 2(2n + 1) > 2(n+1) + 1  
\end{equation*} and so the claim follows by induction.

The above two claims imply that if \(g(n)>f(n)\), then \(g(n+1)>f(n+1)\)
for all \(n>4\). However, \(g(5)>f(5)\). So if \(x=5\), then
\(f(n)\leq g(n)\) for all \(n\geq x\). So \(f\in O(g)\), as desired.

\hypertarget{b-2}{%
\paragraph{3 (b)}\label{b-2}}

Note that \(2^{n+1} = 2*(2^n)\). So if \(c=2\) and \(x=0\), it follows
that that \(2^{n+1}\leq c*2^n\) for all \(n\geq x\).

\hypertarget{c-1}{%
\paragraph{3 (c)}\label{c-1}}

Suppose there were \(c,x\) such that \(2^{2n} \leq c*2^n\) for all
\(n\geq x\). Since \(2^n>0\), it follows from
\(2^{2(n+1)} \leq c * 2^{n+1}\) that \(2^{n+1} \leq c\) for all
\(n\geq x\).

However, \(2^n>n\) for all \(n\geq0\) (as is readily proved by induction
on \(n\)). Hence, \(2^{c+1}> c+1 > c\). We must therefore have \(c<x\),
so that \(c < x < 2^x < 2^{x+1}\), a contradiction.

\hypertarget{d-1}{%
\paragraph{3 (d)}\label{d-1}}

Let \(f(n) = 2n\), and let \(g(n) = n\). Clearly \(f\in O(g)\), since we
may pick \(c=2\). But, writing \(\hat f(n) = 2^{f(n)}\) and
\(\hat g(n) = 2^{g(n)}\), the above response to exercise 3c says
precisely that \(\hat f\not\in O(\hat g)\).

\hypertarget{section}{%
\paragraph{4}\label{section}}

First, note that \(a^{\log_a(x)} = x\), so that \begin{equation}
  \log_a(x)  \log_b(a) = \log_b(a^{\log_a(x)})  = \log_b(x).
\end{equation} Therefore \begin{equation}
  \log_a(x) = \log_b(x)/\log_b(a)
\end{equation} (the ``change of base formula''). So setting
\(c=1/\log_b(a)\) and \(x=1\), it follows that
\(\log_a(x_0)\leq c\log_b(x_0)\) for all \(x_0>x\).

\hypertarget{a-3}{%
\paragraph{5 (a)}\label{a-3}}

I'll argue by induction on \(n\). Clearly \(x^{j}\in O(x^n)\) if
\(n=j\). Now, suppose that \(x^j \leq cx^n\) for all \(x\). Picking
\(\hat x\) to be the greater of \(x\) and \(1\), it follows that
\(x^j \leq cx^n \leq c x^{n+1}\) for all \(x\geq \hat x\), as desired.

Before continuing to (b), note that if \(g\in O(f)\), then
\(O(g)\subseteq O(f)\). Hence part (a) implies that
\(O(x^j)\subseteq O(x^n)\) for all \(j\leq n\).

\hypertarget{b-3}{%
\paragraph{(b)}\label{b-3}}

I'll argue by induction on \(n\). The claim is trivial for \(n=1\).
Writing \(p(x) = \sum_{i=1}^{n}a_nx^n\), suppose that
\(\sum_{i=1}^{n}a_nx^n \in O(x^n)\). It then follows from the above
remark on part (a) that \(p(x) \in O(x^{n+1})\). Meanwhile, clearly
\(a_{n+1}x^{n+1} \in O(x^{n+1})\). The claim follows from the fact that
\(O(x^{n+1})\) is closed under addition (Lemma 3.1 in the handout).

It now suffices to show that \(O(f)\) is closed under addition for any
\(f\). Well if \(O(f)\) contains both \(g\) and \(h\), then there are
\(x_0, x_1\) and \(c_0, c_1\) such that \(g(n)\leq c_0f(n)\) and
\(h(n)\leq c_1f(n)\) for all \(n\) respectively no less than
\(x_0, x_1\). Picking \(c\) and \(x\) as \(c_0+c_1\) and the larger of
\(x_0, x_1\), it follows that \(g(x) + h(x) \leq cf(n)\) for all
\(n \geq x\).

\hypertarget{c-2}{%
\paragraph{(c)}\label{c-2}}

First, note that \(p(x)\geq a_nx^{n}\) for all \(x>0\), so it suffices to
show that \(a_nx^{n}\in \Omega(x^n)\). Since \(a_n>0\), we may simply
pick \(c=a_n\) to affirm that \(a_nx^n \geq cx^n\) for all \(x\) so that
\(a_nx^n\in \Omega(x^n)\) as desired.

\hypertarget{d-2}{%
\paragraph{(d)}\label{d-2}}

Suppose that \(p(x)\in O(x^d)\), where \(p(x) = \sum_{i=1}^na_ix^i\).
Toward a contradiction further suppose that \(a_n>0\). Then likewise
\(a_nx^n\in O(x^d)\), and so, since \(a_n>0\), also \(x^n\in O(x^d)\).
Then there are \(c, \hat x\) such that \(x^n \leq cx^d\) for all
\(x\geq\hat x\). So writing \(m=n-d\), we have \(m>1\) while
\(x^m \leq c\) for all \(x\geq \hat x\). This contradicts the fact that
\(x^m > c\) for any \(x\) greater than both \(1\) and \(c\).

\hypertarget{section-1}{%
\paragraph{6}\label{section-1}}

Suppose that \(f\in O(g)\) and \(g\in O(h)\). Then there are
\(c_0,c_1,x_0, x_1\) such that \(f(x)\leq c_0g(x)\) and
\(g(x)\leq c_1h(x)\) for all \(x\) greater than \(x_0,x_1\)
respectively. Picking \(c\) to be \(c_0c_1\) and \(\hat x\) to be the
larger of \(x_0,x_1\), it follows that \(f(x)\leq ch(x)\) for all
\(x\geq \hat x\).

\hypertarget{a-4}{%
\paragraph{7 (a)}\label{a-4}}

\(T(n) = 2T(\frac{n}{2}) + n^4\). Setting \(a=2\) and \(b=2\), it
follows that \(\log_b a = 1\) so that
\(f(n) = n^4\in O(n^{\log_b a + \epsilon})\) for \(\epsilon=3\).
Furthermore, picking \(c=1\), \begin{equation*}
af\left(\frac{n}{b}\right) = 2\left(\frac{n}{2}\right)^4 = \frac{n^4}{8}\leq n^4 = cf(n)  
\end{equation*} for all sufficiently large \(n\) (e.g., \(>1\)).

So, we are in case (3) of the master theorem, and
\(T(n) = \Theta(f(n)) = \Theta(n^4)\).

\hypertarget{b-4}{%
\paragraph{(b)}\label{b-4}}

\(T(n) = T(\frac{7n}{10}) + n\). Setting \(a=1\) and \(b=\frac{10}{7}\),
it follows that \(\log_ba=\log_{\frac{10}{7}}1 = 0\). So here
\(f(n) = n^1 \in O(n^{\log_b a + \epsilon})\) for \(\epsilon = 1\).
Furthermore, again picking \(c=1\), \begin{equation*}
  f\left(\frac{7n}{10}\right)=\frac{7n}{10}\leq n = cf(n)
\end{equation*} for all \(n\geq 0\).

So we are again in case (3) of the master theorem, and
\[
T(n) = \Theta(f(n))=\Theta(n).
\]

\hypertarget{c-3}{%
\paragraph{(c)}\label{c-3}}

\(T(n) = 16T(\frac{n}{4}) + n^2\). With \(a=16\) and \(b=4\), it follows that
\(f(n)=n^2\in O(n^2)=O(n^{\log_b a})\).

Here we are in case (2) of the master theorem, and
\[
T(n)=\Theta(n^{\log_b a}\lg n)=\Theta(n^2\lg n).
\]

\hypertarget{f}{%
\paragraph{(f)}\label{f}}

\(T(n) = 2T(\frac{n}{4}) + \sqrt n\). With \(a=2\) \(b=4\), it follows that
\(f(n)=\sqrt{n} = n^{\frac{1}{2}}\in O(n^{\frac{1}{2}})=O(n^{\log_b a})\).

So again we are in case (2) of the master theorem, and
\[
T(n)=\Theta(n^{\log_b a}\lg n)=\Theta(\sqrt{n}\lg n).
\]


\hypertarget{g}{%
\paragraph{(g)}\label{g}}

\(T(n) = T(n-2) + n^2\). In this case, the recurrence doesn't (apparently) have a
form suited to the master theorem. However, note for example that
\begin{equation*}
T(7) = T(1) + 3^2 + 5^2 + 7^2  
\end{equation*} and this clearly generalizes: \(T(n)\) is the sum of a
constant plus the squares of all odd numbers from \(3\) up to \(n\).
Geometrically, the sum can be approximated as a pyramid with with base
lengths and height each in some constant ratio to \(n\). So a reasonable
guess is that \(T(n) \in \Theta(n^3)\).

Following the substitution method, I'll now argue first, by induction on
\(n\), that \(T(n)\geq \frac{1}{6}n^3\) for all \(n\geq 3\). For the
basis step, we need to consider both \(n=3\) and \(n=4\), which amounts
to observing that \(3^2+T(1) \geq \frac{1}{6}3^3\) and that
\(4^2+T(1) \geq \frac{1}{6}4^3\). Assuming now that
\(T(m)\geq \frac{1}{6}m^3\) for all \(m<n\), it follows for \(n>4\) that
\begin{align*}
  T(n) &= T(n - 2) + n^2\\
       &\geq \frac{1}{6}(n-2)^3 + n^2\\
       % &= \frac{1}{6}(n^3 - 6n^2 + 12n - 8) + n^2\\
       &= \frac{1}{6}n^3 + 2n - \frac{4}{3}\\
       &\geq \frac{1}{6}n^3.
\end{align*} It remains to find conversely an bound on \(T\). For
simplicity, let's assume that the times \(T(1)\) and \(T(2)\) are
negligible (since \(T\) is unbounded, it will eventually swamp
whatever constants they may be). I'll argue, again by induction on
\(n\), that \(T(n)\leq n^3\) for all \(n\geq 12\). It's easy to verify
that this holds for \(n=12\) and \(n=13\), so suppose that it holds for
all \(m<n\). It follows, certainly for \(n>13\), that \begin{align*}
  T(n) &= T(n - 2) + n^2\\
       &\leq n^3 - 6n^2 + 12n - 8 + n^2\\
       % &= n^3 -5n^2 +12n - 8\\
       &\leq n^3.
\end{align*}

\hypertarget{section-2}{%
\paragraph{8.}\label{section-2}}

{[}See attachment.{]}

\hypertarget{a-5}{%
\paragraph{9 (a)}\label{a-5}}

There exist nonempty sets of points in the plane which contain no
uniquely closest pair. For example, \(\{(0,1),(0,2),(0,3)\}\).

\hypertarget{b-5}{%
\paragraph{(b)}\label{b-5}}

The algorithm takes time proportional to \(n^2\), for all \(n\). To see
this, note that the collection of pairs of elements of an \(n\)-element
set has size proportional to \(n^2\). (If the pairs are ordered pairs,
it is exactly \(n^2\), else it is \(\frac{n(n+1)}{2}\)). Since the
algorithm measures a distance for each pair, it must take at least
\(\sim n^2\) steps. Conversely, it requires only a constant number of
tasks per pair; besides measuring the distance \(d\), it must compare
\(d\) with the current maximum distance \(m\) and in the worst case
replace \(m\) with \(d\).

\hypertarget{section-3}{%
\paragraph{10.}\label{section-3}}

Suppose that the nodes of a binary tree are labelled with the entries of
an array \(A\). Specifically, the node of depth \(0\) is labelled with
the first entry \(A[1]\), the nodes of depth \(1\) with the second and
third entries \(A[2]\) and \(A[3]\), and so on.\footnote{I'm just assuming the fact that there always exists a (unique) such labelling.} The statement in question is this:
\begin{itemize}
\item[(*)] If $N$ is labelled with $A[n]$, then the children of $N$ are labelled with $A[2n]$ and $A[2n+1]$.
\end{itemize}
This statement has counterexamples if the array entries are not all distinct (for example, suppose that the root and its children are labeled $0$, but the children's children are in turn labeled $1$; then the root is labeled with $0=A[2]$, but its children are not labeled with $A[4]$ or $A[4+1]$).

So instead, I will begin by proving a special case of (*), namely where $A[i]=i$ for all $i$.  From this I will derive a weaker version of (*).

Let's say that the index of a given node is the array index of the unique entry by which
it is labelled. The special case to handled is now this:
\begin{itemize}
\item[(A)] Suppose that a given node $N$ has index $n$,
  and that each node of index at most $n$ has two children.
  Then the children of $N$ have indices $2n$ and $2n+1$.
\end{itemize}

Toward a proof of (A), first note two facts.  In both cases, assume that the depth $d$ is less than the depth of $N$ (so that ``all'' nodes of depth $d$ exist).
\begin{itemize}
\item[(B)] The number of nodes of depth $d$ is $2^d$. This is clear for $d=0$.  As for $d+1$, by induction there are $2^d$ nodes of depth $d$, and the nodes of depth $d+1$ are precisely the children of a node of depth $d$.  Since each node has two children, and no two nodes have a child in common, it follows that there are $2(2^d)=2^{d+1}$ nodes of depth $d+1$.
\item[(C)] The number of all nodes of depth less than $d$ is $2^{d}-1$.  This is vacuous for $d=0$. Now, suppose the claim holds for $d$.  The nodes of depth less than $d+1$ are precisely the nodes of depth less than $d$, plus the nodes of depth $d$ itself.  By fact (B), the number of nodes of depth $d$ is $2^{d}$, while by induction hypothesis, the number of nodes of depth less than $d$ is $2^d-1$.  However, $2^{d} + 2^{d} - 1 = 2^{d+1}-1$, which proves the claim.
% \item[(D)] The \(i\)th node of depth \(d\) has index \(2^d+i\). 
% The number of all nodes of depth less than $d$ is $2^d$, provided $d$ is no greater than the depth of $N$.  This is vacuously true for $d=0$.  Now, note that the nodes of depth less than $d$ are precisely the nodes of depth less than $d-1$, together with the nodes of depth $d-1$ itself.  So supposing the number of nodes of depth less than $d-1$ to be $2^{d-1}$, it follows by fact (A) that the number of nodes of depth $d$ is $2^{d-1}+2^{d-1}=2^d$.
\end{itemize}
% Now, let's assume that the nodes of each given depth are ordered ``left to right'', so that if $N,N'$ have the same depth, then $N$ precedes $N'$ iff either they have the same parent and $N$ is its first child, or the parent of $N$ precedes the parent of $N'$.  Each node now has a unique rank $(d,i)$ where $d$ is its depth and $i$ is its (1-indexed) position amongst all nodes of its depth.  Then
% \begin{itemize}
% \item[(D)] The \(i\)th node of depth \(d+1\) has index \(2^d+i\).  The node of rank $(1,1)$ is the first child of the root, so it has index $2=2^{d-1}+i$.  Now suppose the claim holds for all $(d',i')<(d,i)$ with $d>0$.  By (B), the number of nodes of depth at most $d$ is $2^d$, so that the first node of depth $d+1$ has index $2^d+1$.  
% \end{itemize}
Now, suppose that the node $N$ is the $i$th node of depth $d$.  
By (C) above, there are $2^d-1$ nodes of depth less than $d$.
So, $N$ has index $2^d+i-1$.  On the other hand, $N$ has $i-1$ predecessors at depth $d$,
and every predecessor of $N$ has two children.
So, the children of $N$ have $2(i-1)$ and $2(i-1)+1$ predecessors at depth $d+1$.
Again using fact (C), it follows that they have indices
$2^{d+1}-1+2(i-1)+1=2(2^d+i-1)=2n$ and $2^{d+1}-1+2(i-1) + 2=2n+1$ respectively.  This completes the proof of (A).

Finally, let's return to the situation where the nodes are labeled with values of an arbitrary array $A$. Suppose that $N$ is labeled with $A[m]$, and that $n$ is the index of $N$.  By construction of the labeling, we must then have $A[m]=A[n]$.
So, by (A) we get the generalization I promised earlier:
\begin{itemize}
\item[(D)] If node $N$ is labeled with $A[n]$, and if all predecessors of $N$ have two children, then the children of $N$ are labeled with $A[2n']$ and $A[2n'+1]$ for some $n'$ such that $A[n']=A[n]$.
xf\end{itemize}
\end{document}
