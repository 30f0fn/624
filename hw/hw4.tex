% Options for packages loaded elsewhere
\PassOptionsToPackage{unicode}{hyperref}
\PassOptionsToPackage{hyphens}{url}
%
\documentclass[
]{article}
\usepackage{lmodern}
\usepackage{amssymb,amsmath}
\usepackage{ifxetex,ifluatex}
\ifnum 0\ifxetex 1\fi\ifluatex 1\fi=0 % if pdftex
  \usepackage[T1]{fontenc}
  \usepackage[utf8]{inputenc}
  \usepackage{textcomp} % provide euro and other symbols
\else % if luatex or xetex
  \usepackage{unicode-math}
  \defaultfontfeatures{Scale=MatchLowercase}
  \defaultfontfeatures[\rmfamily]{Ligatures=TeX,Scale=1}
\fi
% Use upquote if available, for straight quotes in verbatim environments
\IfFileExists{upquote.sty}{\usepackage{upquote}}{}
\IfFileExists{microtype.sty}{% use microtype if available
  \usepackage[]{microtype}
  \UseMicrotypeSet[protrusion]{basicmath} % disable protrusion for tt fonts
}{}
\makeatletter
\@ifundefined{KOMAClassName}{% if non-KOMA class
  \IfFileExists{parskip.sty}{%
    \usepackage{parskip}
  }{% else
    \setlength{\parindent}{0pt}
    \setlength{\parskip}{6pt plus 2pt minus 1pt}}
}{% if KOMA class
  \KOMAoptions{parskip=half}}
\makeatother
\usepackage{xcolor}
\IfFileExists{xurl.sty}{\usepackage{xurl}}{} % add URL line breaks if available
\IfFileExists{bookmark.sty}{\usepackage{bookmark}}{\usepackage{hyperref}}
\hypersetup{
  hidelinks,
  pdfcreator={LaTeX via pandoc}}
\urlstyle{same} % disable monospaced font for URLs
\setlength{\emergencystretch}{3em} % prevent overfull lines
\providecommand{\tightlist}{%
  \setlength{\itemsep}{0pt}\setlength{\parskip}{0pt}}
\setcounter{secnumdepth}{-\maxdimen} % remove section numbering

\date{24 February 2020}
\title{CS 624 - HW 4}
\author{Max Weiss}
\begin{document}
\maketitle

% Definition A loop in a graph is a path which begins and ends at the same vertex.

% Definition A loop v0 → v1 → v2 → · · · → vk is simple iff
% 1. k ≥ 3 (that is, there are at least 4 vertices on the path), and
% 2. it contains no vertex more than once, except of course for the first and last vertices, which are
% the same, and
% 3. That (first and last) vertex occurs exactly twice.

% Note: I will use $\xrightarrow{*}$ for the transitive closure of $\to$. I will write $\bar P$ for the ``reversal'' of a path $P$; and write $PQ$ for the concatenation (where it exists) of paths $P,Q$.

\paragraph{1.} Toward a contradiction, suppose that some edge occurs more than once in $P$.  Then $P$ must contain a subpath of the form  $ u\to v\to\cdots\to u\to v$.  We must have $u\neq v$, since otherwise, $u$ will occur at least four times in $P$, contradicting the definition of loop-simplicity.  But then, more than one vertex occurs more than once in $P$, and this too contradicts the definition.

% Definition A tree is a connected1 undirected graph in which there are no simple loops.

\paragraph{2.} Suppose that $G$ is a tree in which $x,y$ are distinct vertices.  Since $G$ is connected, $G$ must contain at least one path from $x$ to $y$.  It therefore suffices to show that $G$ contains at most one simple path from $x$ to $y$.  

Toward a contradiction, suppose that $P$ and $Q$ are distinct simple paths from $x$ to $y$ in $G$.  Let $\bar Q$ be the reversal of $Q$, and let $P\bar Q$ be the concatenation of $P$ with $\bar Q$.  I will argue, by induction on the length of $P\bar Q$, that the graph $G$ must contain a simple loop.

Since $P$ and $Q$ are distinct, at least three vertices of $G$ must have occurrences in $P\bar Q$, so that this must have length $\geq 4$.  Assume, for a basis step, that the length is just $4$.  Then  $P\bar Q$ must have one of the forms $x\to z\to y\to x$ or $x\to y\to z\to x$, where $x,y,z$ are distinct.  Each of these is a simple loop itself, so this completes the basis step.

Now suppose the $P,Q$ are distinct simple paths from $x$ to $y$, such that the length of $P\bar Q$ is some $k>4$.  Further suppose this claim to hold for all distinct simple paths $P',Q'$ with same source and target, such that $P'\bar {Q'}$ has length less than $k$.  If $P\bar Q$ is itself a simple loop then we are done, so suppose it is not.  Since $k>4$, it follows from the definition of simple loop that in $P\bar Q$, either (i) $x$ occurs not exactly twice, or (ii) some vertex other than $x$ occurs more than once.  

However, case (i) is impossible:  $x$ cannot occur less than twice since  $P\bar Q$ is a loop; but  $x$ also cannot occur more than twice, since paths $P,Q$ are simple.

So we must be in case (ii), and some vertex $z\neq x$ must occur more than once in $P\bar Q$. By symmetry, we may also assume that $z\neq y$.  Since paths $P,Q$ are simple, $z$ cannot occur more than once in either.  Therefore, $P$ and $Q$ must have forms $P_1P_2$ and $Q_1Q_2$, where $P_1, Q_1$ each are simple paths from $x$ to $z$ and $P_2, Q_2$ are simple paths from $z$ to $y$.  Moreover, since $P\neq Q$, it follows that either $P_1\neq P_2$ or $Q_1\neq Q_2$.  In either case,  $P_i \bar{Q_i}$ has length $<k$, and the result now follows from the induction hypothesis.


\paragraph{Lineages.}  From exercise (2), it follows that for any node $x$ of a rooted tree, there is a unique simple path from the root to $x$.  I will refer to this path the \emph{lineage} of $x$.  

By definition, $y$ is an ancestor of $x$ iff there is a simple path from the root through $y$ to $x$.  It follows that $y$ is an ancestor of $x$ iff $y$ occurs in the lineage of $x$.  

% About the concept of path, I will make the following stipulations.  For $x$ to occur in path $P$ is for $P$ to be a path \emph{through} $x$, which in turn is for $P$ to have the form $P_1P_2$ where $x$ is the last node of $P_1$ and the first node of $P_2$.  

% I will also assume that if $y$ occurs in $P$ and $P=P_1P_2$, then either $y$ occurs in $P_1$ or $y$ occurs in $P_2$. I won't try to prove this, since the proof would depend on the definition of path.  But it should be pretty clear.

\subparagraph{3a.} 
Toward a contradiction, suppose that node $x$ has parents $y$ and $z$.  Then $x$ has lineages $P$ and $Q$ whose penultimate nodes are $y$ and $z$ respectively.  Since $x$ has at most one lineage, we have $P=Q$ and therefore $y=z$.

% \subparagraph{3a.} A vertex in a rooted tree can have at most one parent.
% Suppose that node $x$ has two parents $y$ and $z$.  Then there are simple paths from $r$ through $y$ to $x$ and from $r$ through $z$ to $x$, whose penultimate terms are $y$ and $z$.  Since their penultimate terms are distinct, the paths must be distinct themselves.  Hence, there are distinct simple paths from $r$ to $x$.  This contradicts the result of exercise 2, that there is at most one simple path between any two nodes.

\subparagraph{3b.} By part (a), it suffices to show that every node other than the root has at least one parent.  If $x$ is not the root, then its lineage has the form $\cdots y\to x$ for some $y$, and this $y$ must be a parent of $x$.

\paragraph{4.} Suppose that $x$ is an ancestor of $y$, and vice versa.  Then $x$ has a lineage $P$ through $y$, and $y$ has a lineage $Q$ through $x$.  By definition, $P$ must contain a simple path from $y$ to $x$, and $Q$ must contain a simple path from $x$ to $y$.  By exercise (2) it follows that $x=y$.  

% Then there is a simple path from $r$ through $x$ to $y$, and a simple path from $r$ through $y$ to $x$.  Any subpath of a simple path is a simple path (since whatever occurs more than once in the subpath occurs more than once in the path).  So, there are simple paths from $y$ to $x$ and from $x$ to $y$.  By the result of exercise (2), it follows that $x=y$.  


% \paragraph{4.} Suppose that $x$ is an ancestor of $y$, and vice versa.  Then there is a simple path from $r$ through $x$ to $y$, and a simple path from $r$ through $y$ to $x$.  Any subpath of a simple path is a simple path (since whatever occurs more than once in the subpath occurs more than once in the path).  So, there are simple paths from $y$ to $x$ and from $x$ to $y$.  By the result of exercise (2), it follows that $x=y$.  

% 1.5 Exercise Suppose T is a rooted tree with root r and x \neq r is an vertex in T.
% Further suppose a and b are both ancestors of x. And to make things simple, suppose that a \neq b.
% Prove that either a is an ancestor of b or b is an ancestor of a.

\paragraph{5.}  
Suppose that $x$ has a lineage $P$ through $a$,  and a lineage $Q$ through $b$.  Then $P$ must have the form $P_1P_2$, where $P_1$ is a lineage of $a$, and $P_2$ is a path from $a$ to $x$. Now if $b$ occurred neither in $P_1$ nor in $P_2$, then $b$ would not occur in $P$.  Since $b$ does occur in $Q$ we would then have $P\neq Q$, and $x$ would have two lineages wihch is impossible.  

Therefore, either $b$ occurs in $P_1$ or $b$ occurs in $P_2$. If $b$ occurs in $P_1$, then $P_1$ is a lineage of $a$ through $b$, so that $b$ is an ancestor of $a$. So suppose that $b$ occurs in $P_2$.  Then $P_2$ has an initial segment $Q$ which is a simple path from $a$ to $x$.  Consequently, the concatenation of $P_1$ with $Q$ is a lineage of $b$ through $a$ so that $a$ is an ancestor of $b$.  

% Prove that any two nodes x and y in a rooted tree have a unique least common ancestor
% z. This simply means that there is a path P1 from z to x, and a path P2 from z to y, and that the
% only vertex those two paths have in common is z. (And further, that there is exactly one node z
% with this property.)




\paragraph{6.} Let's say that $z$ is a \emph{common ancestor} of $x,y$ if it is an ancestor of each of $x$ and $y$.  Also, let's say that a node $x$ is \emph{minimal} in a set $X$ of nodes provided that $x$ is the only descendant of $x$ in $X$.  Finally, a \emph{least common ancestor}\footnote{An alternative formulation would be this: $r$ is a least common ancestor of $x,y$ iff $r$ is a least element in the intersection of the lineages of each of $x$ and $y$; this is equivalent because the common ancestry of $x,y$ is just the intersection of their lineages.} of $x,y$ is a node which is minimal in the set of common ancestors of $x,y$.  

To show that any nodes $x,y$ have a unique least common ancestor, it suffices to show three things: (i) $x, y$ have at least one common ancestor; (ii) any nonempty set of nodes has a minimal element; and (iii) the set of common ancestors of $x,y$ has most one minimal element.

Part (i) is clear, since the root is an ancestor of all nodes in the tree. 

% As for part (ii), first note the following lemma.  Where $x,y$ are nodes of a tree, let's write $x\leq y$ to mean that $x$ is a descendant of $y$.  I will argue that $\leq$ is indeed a partial order, i.e., a transitive antisymmetric relation.  Antisymmetry follows from exercise (4).  As for transitivity, suppose that $x\leq y$ and that $y\leq z$.  Then, there are simple paths $P$ from the root through $x$ to $y$, and $Q$ from the root through $y$ to $z$.  Now $Q$ has the form $Q_1Q_2$ where $Q_1$ is a simple path from $r$ to $y$ and $Q_2$ is a simple path from $y$ to $z$.  By exercise (2), we must have that $Q_1=P$.  Hence, $Q_1=P_1P_2$ where $P_1$ is a simple path from $r$ to $x$ and $P_2$ is a simple path from $x$ to $y$.  It follows that $Q=P_1(P_2Q_2)$ is a simple path from $r$ through $x$ to $z$, so that $x$ is indeed an ancestor of $z$.

As for part (ii), recall that a partial order is a transitive antisymmetric relation.  I will argue that if $X$ is a nonempty but finite, partially ordered set, then $X$ has a minimal element.  Suppose, to the contrary, that $X$ does not have a minimal element.  I will argue that $X$ must be infinite, because for all $n$, it contains a descending chain of $n$ distinct elements.  This is clear for $n=1$.  So suppose that $x_1\geq\cdots\geq x_n$ is a chain of $n$ distinct elements.  By hypothesis, $x_n$ is not minimal in $X$, so there must be a $y\in X$ with $y\leq x_n$ and $y\neq x_n$.  If $y=x_i$ for some $i<n$, then by transitivity of $\leq$ we would have $x_n\leq y$ while $y\leq x_n$, so that $y=x_n$ by antisymmetry; and this is a contradiction.

%% I will prove this by induction on the size of $X$.  If this size is $1$, then the claim is trivial.   So suppose it holds for all sets of size at most $k$, and let $X$ have size $k+1$.  Then, there is a set $X'$ of size $k$, and an element $x$ of $X$ such that $X=X'\cup\{x\}$.  By induction hypothesis, let $x'$ be a minimal element of $X'$.  If $x\not\leq x'$, then $x'$ is also minimal in $X$.  So suppose $x\leq x'$.  I will argue that in this case $x$ is minimal in $X$.  Suppose that $z\leq x$; it remains to show that $z=x$.  Well, if $z\neq x$, then $z\in X'$.   Since we are dealing with a transitive relation and $x\leq x'$, it follows that $z\leq x'$.  But since $x'$ is minimal in $X'$, therefore $z=x'$.  We thus have $z\leq x\leq z$ and therefore $z=x$, a contradiction.  

Let's now work to apply this lemma in the case at hand.  Where $x,y$ are nodes of a tree, I'll here write $x\leq y$ to mean that $x$ is a descendant of $y$.  I will argue that $\leq$ is indeed a partial order.  Antisymmetry follows from exercise (4).  As for transitivity, suppose that $x\leq y$ and that $y\leq z$.  Then $y$ has a lineage $P$ through $x$, and $z$ has a lineage $Q$ through $y$.  In particular, $Q$ has the form $Q_1Q_2$, where $Q_1$ is a simple path from the root to $y$, and $Q_2$ is a simple path from $y$ to $z$.  By exercise (2), we must have that $Q_1=P$.  Hence $Q_1=P_1P_2$, where $P_1$ is a lineage for $x$, and $P_2$ is a simple path from $x$ to $y$.  It follows that $Q=P_1(P_2Q_2)$ is a lineage for $z$ through $x$, so that $x$ is indeed an ancestor of $z$.


%% Since $\leq$ is a partial order, (ii) now follows



Finally as for (iii), let $u$ and $v$ each be a common ancestor of $x,y$.  Then each is an ancestor of $x$.  So by exercise (5), it follows that either $u$ is an ancestor of $v$, or $v$ is an ancestor of $u$.  Without loss of generality, assume that $u$ is an ancestor of $v$.  Then there is a simple path from the root through $u$ to $v$; hence either $v=u$, or $v\neq u$ and $v$ is not minimal.  In any case, no distinct common ancestors are minimal, so at most one minimal common ancestor exists.


\paragraph{7.} 

First some remarks about ``order determined by the tree''. In the notes, the successor of a node in a binary search tree is said to be the node which is ``the successor in the order determined by the tree.''  Intuitively, this order is the smallest reflexive relation $R$ such that for every node $x$, all descendants of the left child of $x$ bear $R$ to $x$, while $x$ bears $R$ to all descendants of its right child.

It will be useful to have a somewhat closer analysis. Let $\leq_1$ and $\leq_2$ be relations\footnote{In the set-of-ordered-pairs construal of ``relation''.} on the sets $X_1$ and $X_2$.  I'll write ${\leq_1}\circ{\leq_2}$ for the relation ${\leq_1}\cup {\leq_2}\cup (X_1\times X_2)$ on $X_1\cup X_2$.  Clearly $\circ$ is associative.  It is also clear that if each of ${\leq_1},{\leq_2}$  is a total order and if $X_1,X_2$ are disjoint, then ${\leq_1}\circ{\leq_2}$ is a total order on $X_1\cup X_2$.
We can then define, by induction on nodes $x$, the ordering determined by the (sub-) tree rooted at $x$:
\begin{equation}
  \label{defOrd}
  {\leq_x} \;\;=\;\; {\leq^{L}_x}\;\circ\; \{\langle x,x\rangle\}\;\circ\;{\leq^{R}_x}
\end{equation}
where if $x$ has a left child $y$, then $\leq^L_x$ is $\leq_{y}$, and otherwise is the empty set; similarly for $\leq^R_x$ and the right child of $x$.  Since subtrees rooted at distinct children must be disjoint, $\circ$ preserves the property of being a total order, and so it follows by induction that $\leq_x$ is a total order on the tree rooted at $x$.

The order $\leq_r$ determined by a tree with root $r$ is in turn supposed to generate a notion of ``successor'' of a given node $x$.  Suppose that $x$ is not the largest node in the tree.  Now consider the set of all nodes which are greater than $x$.  By the lemma to exercise 6 this must contain at least one minimal element.  And since indeed $\leq_r$ is a total order, there can be at most one.  I will refer to this least node greater than $x$ as the \emph{successor} of $x$.

%% Let's say an ordered set $\X$ consists of a set $\dom{\X}$ called its domain, together with a total order on $\dom{\X}$.  Suppose that $\X,\Y$ are ordered sets with order relations $\leq_X$ and $\leq_Y$.  Let $\leq$ be a relation defined as follows: $x\leq y$ iff either $x\leq_X y$, or $x\leq_Y y$, or $x\in \dom{\X}$ and $y\in \dom{\Y}$.  It is clear that $\dom{\X}\cup\dom{\Y}$ forms an ordered set with $\leq$; and for this I'll write $\X\circ\Y$. Finally, the order relation determined by a tree can be defined as follows:



The foregoing stuff about order is pure theory of binary trees.  Now a \emph{labeled} binary tree is a binary tree plus a function which maps nodes to keys.  I will write $k_x$ for the key of node $x$.  Finally a \emph{binary search tree} is a labeled search tree such that the key of every node of the left subtree of each node $x$ is no greater than $x$, and the key of every node of the right subtree of $x$ is no smaller than $x$.

In this question we are asked to assume that the tree contains no duplicate keys.  We then get the following useful fact:
\begin{itemize}
\item[(*)] if $y$ is a descendant of $x$, then $y$ is in the left subtree of $x$ iff $k_y<k_x$, and $y$ is in the right subtree of $x$ iff $k_y>k_x$.
\end{itemize}

Another useful fact is this.
\begin{itemize}
\item[(**)] Suppose the key function maps each node to its ordinal position in order determined by the tree itself, as this is defined above.  The result is a binary search tree.
\end{itemize}
In other words, any descendant of the left child of some node $x$ is less than $x$ in the order determined by the tree, and similarly for descendants of the right child. 



\subparagraph{7a.} I will first argue that if $a$ is an ancestor of $x$ and $k_a>k_x$, then $x$ is in the left subtree of $a$.  Since $x\neq a$, therefore $x$ must be in either the left or the right subtree of $x$.  But $x$ cannot be in the right subtree of $a$ since $k_a>k_x$.  Therefore, $x$ is in the left subtree of $a$.  

It is clear that if $d$ is a descendant of $x$ and if $x$ is in the left subtree of $a$, then $d$ is in the left subtree of $a$.  So by fact (*), $k_d<k_a$.

\subparagraph{7b.} 

Suppose that $a$ is the successor of $x$.  Consider the set of all nodes $y$ such that $x\leq_y a$.  By the lemma to exercise (6), this must have a minimal element.  We can now complete the proof by enumerating cases of the definition for $x\leq_y a$.
\begin{itemize}
\item $x\leq_y^L a$, or $x\leq_y^R a$.  This contradicts the minimality of $y$.
\item $x$ descends from the left child of $y$, and $a$ descends from the right.  Then $x<y<a$ and so $a$ is not the successor; hence this is impossible.
\item $x$ descends from the left child of $y$ and $y=a$.  Then $a$ is an ancestor of $x$.
\item $a$ descends from the right child of $y$ and $y=x$.  Then $x$ is an ancestor of $a$.
\end{itemize}

\subparagraph{7c.} Suppose that the right subtree of $x$ is nonempty, so that $x$ has a right child $b$.  Invoking the analysis from (7b), we find that if $x$ descends from the left child of $a$, then $x<b<a$, contradicting the hypothesis that $a$ is the successor of $x$.  Hence, $a$ must descend from the right child of $x$.  Since all descendants of $a$'s right child are greater than $x$, therefore $x$ must be the least of them.

% any descendant of the left child of $x$ is less than $x$... prove this?

It now suffices to prove the following lemma: that the least node $b$ in a tree with root $r$ is not the descendant of any right child.  If to the contrary it were the descendant of some right child $c$, then where $d$ is the parent of $c$ we would have $d <_d z$ for all descendants $z$ of $c$ and in particular $d<_d b$; hence $d<_rb$ which contradicts minimality of $b$.

\subparagraph{7d.} Assume $x$ has an empty right subtree, and that $a$ is the successor of $x$.  Then, the last case of (7b) cannot arise.  So we must be in the penultimate case, where $x$ descends from the left child of $a$.  It remains to show that $a$ is the least node whose left child is an ancestor of $x$.  Suppose to the contrary that $x$ descended from the left child of some $b$, which in turn descends from the left child of $x$.  Well, any descendant of the left child of something is less than it, and so we would have $x<b<a$, contradicting the hypothesis that $a$ is the successor of $x$.


\subparagraph{7e.} Suppose that $x$ has a right child $c$. Then the right subtree of $x$ is nonempty.  So by (7c), the successor $a$ of $x$ is the leftmost descendant of $c$.  Since $a$ is leftmost, it (I suppose by definition) does not have a left child.

\subparagraph{7f.} If $x$ is not the least node in the tree, then a \emph{predecessor} of $x$ is the greatest node less than $x$.  Existence and uniqueness of predecessors is similar to that for successors (using a symmetric form of the lemma for question (6)).  We can then develop a symmetric form of (7b), which in turn can be used to deduce the following counterpart of (7c):  if the left subtree of $x$ is nonempty, then the predecessor of $x$ is just the rightmost node in the left subtree.  Symmetrically with 7e, we conclude that if $x$ has a left child $c$, then the predecessor of $x$ is the rightmost descendant of $c$.


% I will say that a node $x$ is \emph{minimal} in a set $X$ of nodes if for all $y\in X$, we have that if $x$ is an ancestor of $y$ then $y=x$.  Also, I will say that a node $z$ is a \emph{common ancestor} of $x,y$ iff $z$ is an ancestor of each of $x$ and $y$.  On the other hand, I will say (expanding somewhat on the definition given in the text) that $z$ is a least common ancestor of $x,y$ iff there are paths $P$ from $z$ to $x$ and from $z$ to $y$, and $z$ is the only node these paths have in common.

% It suffices to prove two lemmas:
% \begin{itemize}
% \item[(A)] $z$ is a least common ancestor of $x,y$ iff $z$ is minimal in the set of common ancestors of $x,y$
% \item[(B)] For any nodes $x,y$, there is exactly one node $z$ which is minimal in the set of common ancestors of $x,y$
% \end{itemize}

% In one direction for lemma (A), first suppose that $x$ is an ancestor of $y$.  Then $x$ is a common ancestor of $x$ and $y$.  So suppose there is another common ancestor $z$ of $x$ and $y$ such that $x$ is an ancestor of $z$.  By exercise (4), it follows that $z=x$, so that $x$ is minimal in the set of common ancestors.  The case where $y$ is an ancestor of $x$, is similar.

% Continuing with the same direction of lemma (A), let $P_1$ be a path from $z$ to $x$,  and let $P_2$ be a path from $z$ to $y$, such that  $z$ is the only node common to $P_1$ and $P_2$.  



\end{document} 




